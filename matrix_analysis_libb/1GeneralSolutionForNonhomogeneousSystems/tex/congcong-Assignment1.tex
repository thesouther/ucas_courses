\documentclass[UTF8,12pt, a4paper,fleqn]{ctexart}
\usepackage{geometry}
\usepackage{amsmath}
\usepackage{arydshln}
\geometry{left=2.0cm, right=2.0cm, top=1.0cm, bottom=2.5cm}

  \title{Assignment1}
  \author{2019.9.16}
  \date{}
  \begin{document}
  \maketitle
  \subsection*{Exercise.4(a)}
由题非齐次方程可得增广矩阵,并化为行阶梯形式,\\
  \begin{equation*}
  \begin{split}
    A&=\left(
      \begin{array}{c|c}
        \begin{matrix}
          1 & 2 & 1 & 2 \\
          2 & 4 & 1 & 3 \\
          3 & 6 & 1 & 4 
        \end{matrix} & 
        \begin{matrix}
          3 \\
          4 \\
          5
        \end{matrix}
      \end{array}
    \right) \rightarrow
    \left(
      \begin{array}{c|c}
        \begin{matrix}
          1 & 2 & 1 & 2 \\
          0 & 0 & 1 & 1 \\
          3 & 6 & 1 & 4 
        \end{matrix} & 
        \begin{matrix}
          3 \\
          2 \\
          5
        \end{matrix}
      \end{array}
    \right) \rightarrow
    \left(
      \begin{array}{c|c}
        \begin{matrix}
          1 & 2 & 1 & 2 \\
          0 & 0 & 1 & 1 \\
          0 & 0 & 1 & 1 
        \end{matrix} & 
        \begin{matrix}
          3 \\
          2 \\
          2
        \end{matrix}
      \end{array}
    \right)\rightarrow
    \left(
      \begin{array}{c|c}
        \begin{matrix}
          1 & 2 & 0 & 1 \\
          0 & 0 & 1 & 1 \\
          0 & 0 & 0 & 0 
        \end{matrix} & 
        \begin{matrix}
          1 \\
          2 \\
          0
        \end{matrix}
      \end{array}
    \right) \\
    &=E_{[A|b]}
  \end{split}
  \end{equation*}
  由上式可知, x$_2$和x$_4$是自由变量,将其带入原方程可得
  \begin{equation*}
    \begin{split}
      &x_1 =1-2x_2-x_4 \\
      &x_2 \  is \ "free" \\
      &x_3 = 2-x_4 \\
      &x_4 \  is \  "free"
    \end{split}
  \end{equation*}
  由此可得此非齐次线性方程组的通解是:
  \begin{equation*}
  x=\left(\begin{array}{c}
    x_1 \\
    x_2 \\
    x_3 \\
    x_4
  \end{array}
  \right) = \left(
    \begin{array}{c}
      1-2x_2-x_4 \\
      x_2 \\
      2-x_4 \\
      x_4
    \end{array}
  \right)=\left(
    \begin{array}{c}
      1 \\
      0 \\
      2 \\
      0
    \end{array}
  \right)+x_2\left(
    \begin{array}{c}
      -2 \\
      1 \\
      0 \\
      0
    \end{array}
  \right)+x_4\left(
    \begin{array}{c}
      -1 \\
      0 \\
      -1 \\
      1
    \end{array}
  \right)
  \end{equation*} \newline

  \subsection*{Exercise.4(b)}
  由题非齐次方程可得增广矩阵,并化为行阶梯形式,\\
  \begin{equation*}
  \begin{split}
    A=\left(
      \begin{array}{c|c}
        \begin{matrix}
          2 & 1 & 1 \\
          4 & 2 & 1 \\
          6 & 3 & 1 \\
          8 & 4 & 1 
        \end{matrix} & 
        \begin{matrix}
          4 \\
          6 \\
          8 \\
          10
        \end{matrix}
      \end{array}
    \right) \rightarrow
    \left(
      \begin{array}{c|c}
        \begin{matrix}
          2 & 1 & 1 \\
          0 & 0 & 1 \\
          0 & 0 & 2 \\
          0 & 0 & 3 
        \end{matrix} & 
        \begin{matrix}
          4 \\
          2 \\
          4 \\
          6
        \end{matrix}
      \end{array}
    \right) \rightarrow
    \left(
      \begin{array}{c|c}
        \begin{matrix}
          2 & 1 & 0 \\
          0 & 0 & 1 \\
          0 & 0 & 0 \\
          0 & 0 & 0 
        \end{matrix} & 
        \begin{matrix}
          2 \\
          2 \\
          0 \\
          0
        \end{matrix}
      \end{array}
    \right)
    =E_{[A|b]}
  \end{split}
  \end{equation*}
  由上式可知, y是自由变量,将其带入原方程可得
  \begin{equation*}
    \begin{split}
      &x =\frac{1}{2}(2-y) \\
      &y \  is \ "free" \\
      &z = 2
    \end{split}
  \end{equation*}
  由此可得此非齐次线性方程组的通解是:
  \begin{equation*}
  x=\left(\begin{array}{c}
    x \\
    y \\
    z
  \end{array}
  \right) = \left(
    \begin{array}{c}
      1-\frac{1}{2}y \\
      y \\
      2
    \end{array}
  \right)=\left(
    \begin{array}{c}
      1 \\
      0 \\
      2 
    \end{array}
  \right)
  +y\left(
    \begin{array}{c}
      -\frac{1}{2} \\
      1 \\
      0 
    \end{array}
  \right)
  \end{equation*}

\end{document}
